\documentclass{article}
\usepackage[utf8]{inputenc}
\usepackage[catalan]{babel}

\usepackage{amsthm, amssymb, amsmath}

\newtheorem{theorem}{Teorema}[section]%chapter
\newtheorem{proposition}[theorem]{Proposició}

\theoremstyle{definition}
\newtheorem{definition}[theorem]{Definició}
\newtheorem{example}[theorem]{Exemple}
\newtheorem{writing}[theorem]{Escriptura}

\newcommand{\R}{\mathbb{R}}
\newcommand{\Z}{\mathbb{Z}}
\newcommand{\U}{\mathcal{U}}

\usepackage{enumitem}

\begin{document}
\tableofcontents

% Ara per ara, no tinc cap perfil per pujar la informacio que obtinc, aixi que ho pujare tot a sac. Esperem
% rapidament millorar aquests defecte que aparentment detectem
\section{Presentació}
\subsection{Respecte $\mathbb{R}$}
\begin{definition}[$a$ adherent]
$\exists (X_n)$ succesió de punts tal que $X_n \to a$
\end{definition}
\begin{example}
$(a, b)$, $a$ és adherent per la successió $X_n = a + \frac{1}{n} \quad \forall n$
\end{example}
\begin{definition}[Clausura i interior]
\begin{itemize}
\item $[a, b]$ interval tancat
\item $(a, b)$ interval obert
	\begin{itemize}
	\item[Definició:] $(a, b)$ és l'interior de $[a, b]$
	\item[Definició:] $[a, b]$ és la clausura de $(a, b)$
	\end{itemize}
\end{itemize}
\end{definition}

\subsection{Respecte $\R^n$}
\begin{definition}[Bola]
\begin{itemize}
\item[Bola oberta:] $B_r (x) = \{y \in \R^n : ||y -x|| < r\}$
\item[Bola tancada:] $B_r (x) = \{y \in \R^n : ||y -x|| \leq r\}$
\end{itemize}
\end{definition}

\begin{writing}[$\U, A$]
\begin{itemize}
\item $A$, tancat
\item $\U$, obert
\end{itemize}
Aquesta és la tendència
\end{writing}

\begin{definition}[obert i tancat]
\begin{itemize}
\item[obert] $\forall x \in \U, \quad \exists r > 0$ tal que $B_r(x) \subset \U$
\item[tancat] $\forall$ successió $(X_n)$ amb $X_n \in A$ que convergeix a un punt $a$ es compleix $a \in A$
\end{itemize}
\end{definition}

\section{Espais mètrics}
Generalitzem, admet distància
\begin{definition}[Distància]
en un conjunt $X$ és una aplicació
$$d:X*X\to\R$$
\begin{enumerate}[label=\bfseries D\arabic*)]
\item $d(x, y) \geq 0$
\item $d(x, y) = 0 \Leftrightarrow x = y$
\item $d(x, y) = d(y, x)$
\item $d(x, z) \leq d(x, y) + d(y, z)$
\end{enumerate}
\end{definition}

\begin{definition}[Espai mètric]
$(X, d)$ Conjunt $X$ amb l'aplicació distància $d$. On la distància respecta les 4 propietats.
\end{definition}

\begin{example}
\begin{itemize}
\item $(X_i, d_i)$ espais mètrics, $d(x, y) = \max{d_j(x_i, y_i)}$ distància producte
\item $g, f: [0, 1] \to \R$ contínua'inues. $d(f, g) = sup\{|g(x) - f(x)|\}$ distància suprem
\item $X = \Z, p \in$ primer $d(m, n) = \left\{\begin{array}{ll}0 &\text{si } m=n\\p^{-v_p(m-n)} &\text{si } m\neq n\end{array}\right.$ distància $p$-àdica
	\subitem Valoració peàdica $v_p(a) = \max\{k\in \Z| k \geq 0, p^k$ divideix $a \} \forall a \in Z$
\item $d(x,y) = \left\{\begin{array}{ll}0&\text{si }x=y\\1&\text{si }x\neq y\end{array}\right.$
\end{itemize}
\end{example}

\begin{definition}[boles]
Donat $(X, d)$ espai mètric, $p \in X, r > 0$
\begin{itemize}
\item[oberta] $B_r(p) = \{y \in X | d(p, y) < r\}$
\item[tancada] $\bar{B_r}(p) = \{y \in X | d(p, y) \leq r\}$
\item[esfera] $S_r(p) = \{y \in X | d(p, y) = r\}$
\end{itemize}
\end{definition}

\begin{definition}[obert]
\begin{itemize}
\item $(X, d)$ espai mètric
\item $\U \subset X$ és obert
	\begin{itemize}
	\item[si] $\forall p \in \U, \exists r > 0$ tal que
		\begin{itemize}
		\item $B_r (p) \subset \U$
		\end{itemize}
	\end{itemize}
\end{itemize}
\end{definition}

\begin{proposition}
Les boles obertes són obertes.
\end{proposition}
\begin{proof}
Donat
\begin{itemize}
\item $B_s (x)$
\item $x \in X$
\item $s > 0$
	\begin{itemize}
	\item Escollim
	\item $p \in B_s (x)$
	\item $\delta = d(p, x)$
	\item $r = s - \delta$
		\subitem $r \neq 0$ ja que si $d(p, x) = s$, és una contradicció perquè $p \in B_s(x)$
	\end{itemize}
\end{itemize}
Comprovem
$$B_r (p) \overset{?}{\subset} B_s(x)$$
Per a comprovar-ho
\begin{itemize}
\item $q \in B_r (p)$
\item $q \overset{?}{\in} B_s (x)$
	\begin{itemize}
	\item $d(q, x) \leq d(q, p) + d(p, x) < r + \delta = s \Rightarrow q \in B_s(x)$
	\end{itemize}
\end{itemize}
\end{proof}

\end{document}

Per primer cop ho faig amb un pelet de sentit
http://tex.stackexchange.com/questions/45817/theorem-definition-lemma-problem-numbering
